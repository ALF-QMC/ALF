% Copyright (c) 2020-2025 The ALF project.
% This is a part of the ALF project documentation.
% The ALF project documentation by the ALF contributors is licensed
% under a Creative Commons Attribution-ShareAlike 4.0 International License.
% For the licensing details of the documentation see license.CCBYSA.

% !TEX root = doc.tex
%------------------------------------------------------------
\subsection{Sequential updating schemes}\label{sec:updating}
%------------------------------------------------------------
%

The program allows for different types of updating schemes, which are described below and summarized in Tab.~\ref{table:Updating_schemes}.  Except for Langevin dynamics, for a given configuration $C$, we propose a new one, $C'$, with a given probability $T_0(C \rightarrow C')$  and accept it according to   the  Metropolis-Hastings   acceptance-rejection probability, 
\begin{equation}
	P(C \rightarrow C') =  \text{min}  \left( 1, \frac{T_0(C' \rightarrow C) W(C')}{T_0(C \rightarrow C') W(C)} \right),
\end{equation}
to guarantee the stationarity condition.  Here, $ W(C) = \left| \Re \left[ e^{-S(C)} \right] \right| $.

Predicting how efficient a certain Monte Carlo update scheme will turn out to be for a given simulation is very hard, so one must typically resort to testing to find out which option produces best results.   Methods to optimize the acceptance of  global moves include Hybrid Monte Carlo  \cite{Duane85} as well as self-learning techniques  \cite{LiuJ17,Xu17a}.      Langevin dynamics stands apart, and as we will see does not depend on the Metropolis-Hastings   acceptance-rejection scheme.



\begin{table}[h]
	\begin{tabular}{@{} p{0.22\columnwidth} p{0.11\columnwidth} p{0.61\columnwidth} @{}}
		\toprule
		Updating schemes             & Type             & Description \\
		\midrule
		\texttt{Sequential}          & \texttt{logical} & (internal variable) If true, the configurations moves through sequential, single spin flips  \\
		\texttt{Propose\_S0}         & \texttt{logical} & If true, proposes sequential local moves according to the probability $e^{-S_0}$, where $S_0$ is the free Ising action. This option only works for 
		  \texttt{type=1}  operator where the field corresponds to an Ising variable  \\
		\texttt{Global\_tau\_moves}  & \texttt{logical} & Whether to carry out  global moves on a single time slice.
		For a given time slice the user can define which part of the operator string is to be computed sequentially. This is specified by the  variable  \texttt{N\_sequential\_start} and \texttt{N\_sequential\_end}. A number of   \texttt{N\_tau\_Global} user-defined global moves on the given time slice  will then be carried out   \\
		\texttt{Global\_moves}       & \texttt{logical} & If true, allows for global moves in space and time.   A user-defined number \texttt{N\_Global} of global moves in space and time  will be carried out at the end of each sweep \\
		\texttt{Langevin}            & \texttt{logical} & If true, Langevin dynamics is used exclusively (i.e., can only be used in association with tempering) \\
		\texttt{Tempering}           & Compiling option & Requires MPI and runs the code in a parallel tempering mode, also see Sec.~\ref{Parallel_tempering.sec},~\ref{sec:compilation} \\
		\bottomrule
	\end{tabular}
	\caption{Variables required to control the updating scheme. Per default the program carries out sequential, single spin-flip sweeps, and logical variables are set to \texttt{.false.}}
	\label{table:Updating_schemes}
\end{table}

\begin{table}[h]
	\begin{tabular}{@{} p{0.22\columnwidth}  p{0.35\columnwidth}  p{0.35\columnwidth} @{}}
		\toprule
		Updating schemes                          & Variables & Subroutines \\
		\midrule
		\texttt{Sequential}     &     & S0   \\
		\texttt{Propose\_S0}   &      &  S0 \\
		\texttt{Propose\_MALA}   & Delta\_t\_MALA\_sequential, Max\_Force\_MALA\_sequential    &  S0, Ham\_Langevin\_HMC\_S0\_single \\
		\texttt{Global\_tau\_moves}  & N\_Global\_tau, Nt\_sequential\_start, Nt\_sequential\_end     & Global\_move\_tau, Overide\_global\_tau\_sampling\_parameters \\
		\texttt{Global\_tau\_MALA\_moves} & N\_Global\_tau\_MALA, Nt\_sequential\_start, Nt\_sequential\_end, Delta\_t\_MALA\_global\_tau,   MAX\_Force\_MALA\_global\_tau  & Global\_MALA\_move\_tau, Overide\_global\_tau\_sampling\_parameters, Ham\_Langevin\_HMC\_S0\_single \\
		\texttt{Global\_moves}       & N\_Global  & Global\_move, Delta\_S0\_global \\
		\texttt{Langevin}            & Delta\_t\_Langevin\_HMC, Max\_Force   & Ham\_Langevin\_HMC\_S0  \\
		\texttt{HMC}                  & N\_HMC\_sweeps, Delta\_t\_Langevin\_HMC, Leapfrog\_steps     & Delta\_S0\_global, Ham\_Langevin\_HMC\_S0  \\
		\texttt{MALA}                & N\_MALA\_sweeps, Delta\_t\_MALA\_global, Max\_Force\_MALA\_global    &  Global\_MALA\_move, Delta\_S0\_global, Ham\_Langevin\_HMC\_S0 \\
		\texttt{Tempering}     &      & Delta\_S0\_global  \\
		\bottomrule
	\end{tabular}
	\caption{Variables required to control the updating scheme. Per default the program carries out sequential, single spin-flip sweeps, and logical variables are set to \texttt{.false.}}
	\label{table:Updating_schemes}
\end{table}
% 

%------------------------------------------------------------
\subsubsection{Space-time sequential single spin flips: discrete fields}
%------------------------------------------------------------
%d
\label{sec:sequential}
%If the boolean variable \texttt{Sequential\_Sweeps}  is set to true, then  the 
The program adopts per default a sequential, single spin-flip strategy. It will visit sequentially each HS field in the space-time operator list and  propose a spin flip. Consider  the Ising spin $s_{i,\tau}$. By default (\texttt{Propose\_S0=.false.}), we will flip it with probability $1$, such that for  this local move  the  proposal matrix is symmetric.  If we are considering the HS field $l_{i,\tau}$  we will propose with probability $1/3$ one  of the other three  possible fields.    For a continuous field, we  modify it with a box distribution of width \texttt{Amplitude}  centered around the origin.   The default value of  \texttt{Amplitude} is set to unity. These updating rules are defined in the   \texttt{Fields\_mod.F90} module  (see Sec.~\ref{sec:fields}). Again, for these local moves, the proposal matrix is symmetric.  Hence in all cases we will accept or reject the move according to 
\begin{equation}
P(C \rightarrow C') =  \text{min}  \left( 1, \frac{ W(C')}{W(C)} \right).
\end{equation}

This default updating scheme can be overruled by, e.g., setting \texttt{Global\_tau\_moves} to \texttt{.true.} and not setting \texttt{Nt\_sequential\_start} and \texttt{Nt\_sequential\_end} (see Sec.~\ref{sec:input}).
It is also worth noting that this type of sequential spin-flip updating does not satisfy detailed balance, but rather the more fundamental stationarity condition \cite{Sokal89}. 


% 
%------------------------------------------------------------
\paragraph{Sampling of $e^{-S_0}$}
\label{sec:S0}
%------------------------------------------------------------
% 
The package can also propose single spin-flip updates according to a non-vanishing free bosonic 
action $S_0(C)$. This sampling scheme is used if the logical variable \texttt{Propose\_S0} is set to \texttt{.true.}.   
As mentioned previously, this option only holds for Ising variables. 

Consider an Ising spin at space-time $i,\tau$ in the configuration $C$. Flipping this spin generates the configuration $C'$ and we propose this move according to 
\begin{equation}
T_0(C \rightarrow C')  =  \frac{e^{-S_0(C')}}{ e^{-S_0(C')} + e^{-S_0(C)} }   = 1 - \frac{1}{1 +  e^{-S_0(C')} /e^{-S_0(C)}}.
\end{equation}
Note that the function $\texttt{S0}$ in the  \texttt{Hamiltonian\_Hubbard\_include.h}  module  computes precisely the ratio\\
 ${e^{-S_0(C')} /e^{-S_0(C)}}$, therefore $T_0(C \rightarrow C') $ is obtained without any additional calculation. 
The proposed move is accepted with the probability: 
\begin{equation}
 P(C \rightarrow C') =  \text{min}  \left( 1,  \frac{e^{-S_0(C)}   W(C')}{ e^{-S_0(C')} W(C)} \right).
\end{equation}
Note that, as can be seen from Eq.~\eqref{eqn:partition_2}, the bare action $S_0(C)$  determining the  dynamics of the bosonic configuration in the absence of coupling to the fermions does not enter the Metropolis acceptance-rejection step.
% 
% 

As mentioned above,  this sequential updating scheme is space-time is the default and  requires the following settings in the 

\begin{lstlisting}[style=fortran,escapechar=\#,breaklines=true]
&VAR_QMC 
......
Sequential           = .T.  ! Flag for sequential moves
Propose_S0           = .F.  ! Whether to propose moves according to e^{-S0}
......
/
\end{lstlisting}
%------------------------------------------------------------

%------------------------------------------------------------
\subsubsection{Space-time sequential single spin flips: Continuous fields }
%------------------------------------------------------------
%
\label{sec:sequential_continuous}
If the  field is a continuous real variable,  \texttt{type=3}, (See Sec.~\ref{sec:fields}) then  the program  provides two  types of  updating schemes.

\paragraph{Box distribution} The first one is the default and consists in proposing a new value for the field according to a box distribution of width \texttt{Amplitude} centered around the origin.  This updating scheme is symmetric, and the move is accepted with probability $P(C \rightarrow C') =  \text{min}  \left( 1, \frac{ W(C')}{W(C)} \right)$. This is the default updating scheme for continuous fields. 

\paragraph{Metropolis assisted Langevin algorithm (MALA)} \label{sec:mala_sequential} 
This updating scheme is based on the Langevin equation  that we will discuss in Sec.~\ref{sec:langevin}.  For  a given field $s_{n,\tau}$ (See. Eq.~\ref{Btau.eq}), we propose a new value of the field according to a single step of the discretized Langevin equation just for this field: 
\begin{equation}
	s_{n,\tau}' = s_{n,\tau} - \frac{\partial S}{\partial s_{n,\tau}} \delta t + \sqrt{ 2 \delta t} \eta
\end{equation}
where $\eta$ is a Gaussian random variable with zero mean and unit variance, and $\delta t$ is the time step of the Langevin dynamics.  This updating scheme is not symmetric, and the move is accepted with probability  the Metropolis-Hastings acceptance-rejection probability,
\begin{equation}P(C \rightarrow C') =  \text{min}  \left( 1, \frac{T_0(C' \rightarrow C) W(C')}{T_0(C \rightarrow C') W(C)} \right),\end{equation}
where $T_0(C \rightarrow C')$ is the probability to propose the move $C \rightarrow C'$ and reads:
\begin{equation}
	T_0(C \rightarrow C') = \frac{1}{\sqrt{2 \pi \delta t}} e^{-\frac{(s_{n,\tau}' - s_{n,\tau} + \frac{\partial S}{\partial s_{n,\tau}} \delta t)^2}{2 \delta t}}.
\end{equation}
The user can control the time step $\delta t$ by setting the variable \texttt{Delta\_t\_MALA\_sequential} in the input file.  The maximum value of the force $\frac{\partial S}{\partial s_{n,\tau}}$ is controlled by the variable \texttt{Max\_Force\_MALA\_sequential}, which is used to avoid proposing moves with very low acceptance.  The force is then defined as $\text{sign}(\frac{\partial S}{\partial s_{n,\tau}}) \text{min}(\left|\frac{\partial S}{\partial s_{n,\tau}}\right|, \texttt{Max\_Force\_MALA\_sequential})$.


\begin{lstlisting}[style=fortran]
&VAR_QMC
......
Sequential           = .T.       ! Flag for sequential moves
Propose_S0           = .F.       ! Whether to propose moves according to e^{-S0}
Amplitude            = 0.5D0     ! Amplitude of the box distribution
Propose_MALA         = .true.    ! MALA sequential moves for continuous fields
Max_Force_MALA_sequential  = 1.0 ! Maximum value of the force 
Delta_t_MALA_sequential    = 0.1 ! Time step for the MALA 
......
/
\end{lstlisting}

We note that we have two routines to compute the force of the bosonic part of the action, \texttt{S0}.  For the  Langevin dynamics  described in Sec.~\ref{sec:langevin} the force for  each  field  has to be computed. This is done in the routine \texttt{Ham\_Langevin\_HMC\_S0}  that returns the two-dimension array 
of forces. For the MALA sequential moves, we only need the force for a single field, which is computed in the routine \texttt{Ham\_Langevin\_HMC\_S0\_single}.   If this routine is not provided by the user, the code will call \texttt{Ham\_Langevin\_HMC\_S0} and extract the force for the given field.  This is computationally expensive,  such that a warning is printed on the standard output at execution. The following routine must be provided by the user before this option can be used: 
\begin{lstlisting}[style=fortran]
Subroutine Ham_Langevin_HMC_S0(Forces_0)
  Real (Kind=Kind(0.d0)), Intent(inout), allocatable :: Forces_0(:,:)
  ! Forces_0(n,nt) = \frac{\partial S_0}{\partial s_{n,\tau}}.  
  ! Here n is the index of the field in the operator list and nt is the time slice.	
End Subroutine Ham_Langevin_HMC_S0
\end{lstlisting}

%------------------------------------------------------------

%------------------------------------------------------------

\subsubsection{Time sequential, global in space updates.}
\paragraph{User-defined global moves on a single time slice}
\label{sec:global_space}
This option allows one to carry out  user-defined global moves on a single time slice.  This option is enabled by setting the logical variable  \texttt{Global\_tau\_moves} to \texttt{.true.}.  Recall that the propagation over a time step $\Delta \tau$   (see Eq. \ref{Btau.eq}) can be  written as: 
\begin{equation}
	e^{-V_{M_I+M_V}(s_{M_I+M_V,\tau})}  \cdots e^{-V_{1}(s_{1,\tau})}  \prod_{k=1}^{M_T}   e^{-\Delta \tau {\bm T}^{(k)}},
\end{equation}
where $e^{-V_{n}(s_{n})}$ denotes one element of the  operator list  containing the HS fields.  One can provide  an interval of indices, 
\texttt{[Nt\_sequential\_start, Nt\_sequential\_end]},  in which the operators will be updated  sequentially. Setting \texttt{Nt\_sequential\_start}$\; = 1$ and \texttt{Nt\_sequential\_end}$\; = M_I+M_V$  reproduces the  sequential single spin flip strategy of the above section.

The variable \texttt{N\_Global\_tau}  sets the number of global moves carried out on each time slice \texttt{ntau}. Each global move is generated in the routine  \texttt{Global\_move\_tau}, which is provided by the user in the Hamiltonian file. In order to  define this move, one specifies the following variables: 
\begin{itemize}
\item \texttt{Flip\_length}:  An integer stipulating the  number of spins to be flipped.
\item \texttt{Flip\_list(1:Flip\_length)}:   Integer array containing the  indices of the operators to be flipped.
\item \texttt{Flip\_value(1:Flip\_length)}:  \texttt{Flip\_value(n)} is an  integer containing the new value of the  HS  field for the operator \texttt{Flip\_list(n)}.
\item  \texttt{T0\_Proposal\_ratio}:   Real number containing  the quotient
\begin{equation}
	 \frac{T_0(C' \rightarrow C)}{T_0(C \rightarrow C') }  \;, \label{T0_ratio}
\end{equation}
where $ C'$  denotes the new configuration  obtained by flipping the spins specified in the \texttt{Flip\_list}  array. 
Since we allow for a stochastic  generation of  the global move, it may very well be that no change is proposed. In this case, \texttt{T0\_Proposal\_ratio}   takes the value 0 upon exit of the routine \texttt{Global\_move\_tau} and  no update is carried out. 
\item \texttt{S0\_ratio}:   Real number containing  the ratio  $e^{-S_0(C')}/e^{-S_0(C)}$. 
\end{itemize}
Specifically, the routine \texttt{Global\_move\_tau} must be provided by the user and has the following structure:
\begin{lstlisting}[style=fortran]
Subroutine Global_move_tau(T0_Proposal_ratio, S0_ratio, &
        &                       Flip_list, Flip_length,Flip_value,ntau)

    Real (Kind = Kind(0.d0)),   INTENT(OUT) :: T0_Proposal_ratio,  S0_ratio
    Integer                   , INTENT(OUT) :: Flip_list(:)
    Complex (Kind = Kind(0.d0)),INTENT(OUT) :: Flip_value(:)
    Integer, INTENT(OUT) :: Flip_length
    Integer, INTENT(IN)  :: ntau
             
end Subroutine Global_move_tau
\end{lstlisting}
The subroutine \texttt{Overide\_global\_tau\_sampling\_parameters} can be used to overrule the default values of the parameters controlling 
the global moves on a single time slice, which are set in the input file.  
\begin{lstlisting}[style=fortran]
Subroutine Overide_global_tau_sampling_parameters(Nt_sequential_start,Nt_sequential_end,&
                                        &         N_Global_tau, N_Global_tau_MALA)
	Integer, Intent(OUT) :: N_Global_tau, Nt_sequential_start, Nt_sequential_end
end Subroutine Overide_global_tau_sampling_parameters
\end{lstlisting}
%------------------------------------------------------------
As an example, one can run the Hubbard Hamiltonian with the parameters: 
\begin{lstlisting}[style=fortran]
&VAR_QMC
......
Sequential       = .T.       ! Flag for sequential moves
Global_tau_moves = .T.       ! Global moves on a time slice
N_Global_tau = 1             ! Number of global moves will be set 
                             ! to the length of OP_V by the override routine
......
\end{lstlisting}
Here we randomly chose  a  vertex and update the field.

%------------------------------------------------------------
\paragraph{Metropolis assisted Langevin algorithm (MALA) on a single time slice}
This option will carry out a global move on a single time slice using the Metropolis assisted Langevin algorithm.  To use this option, the 

\begin{lstlisting}[style=fortran]
&VAR_QMC
......
Sequential              = .T.       ! Flag for sequential moves
Global_tau_MALA_moves   = .T.       ! Global MALA moves on a single time slice
N_Global_tau_MALA       = 1         ! Number of global MALA moves will be set 
Delta_t_MALA_global_tau =  0.1      ! Time step for the MALA global-tau move 
Max_Force_MALA_global_tau = 1.0     ! Maximum force for the MALA global-tau move 
......
\end{lstlisting}
The user can override the number of \texttt{N\_Global\_tau\_MALA}  by using the  aforementioned 
\texttt{Overide\_global\_tau\_sampling\_parameters} routine. The  routine:

\begin{lstlisting}[style=fortran]
subroutine Global_MALA_move_tau(Flip_list, Flip_length,ntau)
    Integer                   , INTENT(OUT) :: Flip_list(:)
    Integer, INTENT(OUT) :: Flip_length
    Integer, INTENT(IN)  :: ntau
      
end Subroutine Global_MALA_move_tau
\end{lstlisting}
Allows to  specify the list of field to be updated in the MALA move. By default, all the  field on  the time slice are visited and 
a warning is issued on the standard output. 

\subsubsection{Combining various sequential updating schemes.}
The user can combine the various sequential updating schemes described above.  For  example one can carry out a number of sequential single spin-flip sweeps,  followed 
by a number of global ones.  The range of the sequential sweeps are specified by the variables \texttt{Nt\_sequential\_start} and \texttt{Nt\_sequential\_end}, while the number of global moves is set by \texttt{N\_Global\_tau}.
For  example, the parameter file: 
\begin{lstlisting}[style=fortran]
&VAR_QMC
......
Sequential           = .T.  ! Sequential moves
Propose_S0           = .F.  ! Proposes single spin flip moves with probability exp(-S0) 
Nt_sequential_start   = 1   ! Starting index of the sequential moves on a time slice
Nt_sequential_end     = 3   ! Ending index of the sequential moves on a time slice
Propose_MALA         = .T.  ! MALA sequential moves for continuous fields
Max_Force_MALA_sequential  = 1.0 ! Max force for the MALA sequential moves
Delta_t_MALA_sequential    = 0.1 ! Time step for the MALA sequential moves
Global_tau_moves = .true.    ! Global moves on a time slice
N_Global_tau = 6             ! Number of global moves on a time slice
......
\end{lstlisting}
Will carry out sequential single spin-flip moves on the first three operators of the operator list using the MALA updating scheme \ref{sec:mala_sequential}, followed by \texttt{N\_Global\_tau=6} global moves on the time slice.


\subsection{Global updates in time and space}


\label{sec:global_slice}
%------------------------------------------------------------
%  

\subsubsection{User specified global moves in space and time}
The code allows for global updates as well. The user must then provide two additional functions \texttt{Global\_move} and \texttt{Get\_\-Delta\_\-S0\_\-global( Nsigma\_old )}.   

The subroutine  \texttt{Global\_move(T0\_Proposal\_ratio,nsigma\_old,size\_clust)}  proposes  a global move. 
Its single input is the variable \texttt{nsigma\_old} of type \texttt{Field} (see Section~\ref{sec:fields}) that contains  the full  configuration $C$ stored in \texttt{nsigma\_old\%f(M\_V + M\_I, Ltrot)}.  On output, the new configuration $C'$, determined by the user,  is stored in the two-dimensional array \texttt{nsigma}, which is a global variable declared in the Hamiltonian module.
Like for the global move in space (Sec.~\ref{sec:global_space}), \texttt{T0\_Proposal\_ratio} contains the proposal ratio $\frac{T_0(C' \rightarrow C)}{T_0(C \rightarrow C')} $.
Since we allow for a stochastic  generation of  the global move, it may very well be that no change is proposed. In this case, \texttt{T0\_Proposal\_ratio}   takes the value 0 upon exit, and  \texttt{nsigma$\,=\,$nsigma\_old}.   
The real-valued \texttt{size\_clust} gives the size of the proposed move $\left(\text{e.g.,} \tfrac{\text{Number of flipped spins}}{\text{Total number of spins}}\right)$. This is used to calculate the average sizes of proposed and accepted moves, which are printed in the \texttt{info} file. The variable \texttt{size\_clust} is not necessary for the simulation, but may help the user to estimate the effectiveness of the global update.

In order to compute the acceptance-rejection ratio,  the user must also provide a function 
\texttt{Get\_\-Delta\_\-S0}\texttt{\_\-global(nsigma\_old)} that computes the difference $\Delta S_0=S_0(C)-S_0(C')$. Again, the configuration $C'$ is given by the field \texttt{nsigma}.

The variable \texttt{N\_Global} determines the number of global updates performed per sweep. Note that global updates are expensive, since they require a complete recalculation of the weight.

Practically one will combine global update with local ones. In the implementation  of the Hubbard model,   the parameter file: 
\begin{lstlisting}[style=fortran]
&VAR_QMC
......
Sequential       = .T.       ! Flag for sequential moves
Global_moves     = .T.       ! Flag for global moves in space and time
N_Global         = 6         ! Number of global moves in space and time
......
/
\end{lstlisting}
will carry out 6 global moves in space and time at the end of each sweep. A  possible choice for the global move is to randomly choose a 
lattice point and change the sign of the field at this point for all time slices.  The routine that will do this global update reads:
\begin{lstlisting}[style=fortran]
Subroutine Global_move(T0_Proposal_ratio, nsigma_old, size_clust)
   Real (Kind = Kind(0.d0)),   INTENT(OUT) :: T0_Proposal_ratio, size_clust
   Type(Field), INTENT(IN)  :: nsigma_old

   size_clust = Ltrot
   n  = nranf(Size(Op_V,1))
   nsigma%f(n,:)   = - nsigma_old%f(n,:) 
   T0_Proposal_ratio = 1
end Subroutine Global_move
\end{lstlisting}


% 

%------------------------------------------------------------
\subsubsection{Parallel tempering } 
\label{Parallel_tempering.sec}
%------------------------------------------------------------
% 
Exchange Monte Carlo \cite{Hukushima96}, or parallel tempering \cite{Greyer91}, is a possible route to overcome sampling issues in parts of the parameter space.
Let $h$ be a parameter which one can vary without  altering the configuration space $ \{C  \}  $ and let us assume that for some values of $h$ one encounters sampling problems.   For example, in the realm of spin glasses, $h$  could correspond to the  inverse temperature.  Here at high temperatures the phase space is easily sampled, but at low temperatures  simulations get stuck in local minima. For quantum systems, $h$ could   trigger a quantum phase transition where  sampling issues are encountered, for example, in the ordered phase and not in the disordered one.   As its name suggests, parallel tempering  carries out in parallel simulations at consecutive  values of  $h$:  $h_1, h_2,  \cdots h_n$, with  $h_{1} < h_2 < \cdots < h_n$.  One will sample the extended ensemble:
\begin{equation}
	P(\left[h_1,C_1\right], \left[h_2,C_2\right], \cdots, \left[h_n,C_n\right] ) =  \frac{W(h_1,C_1) W(h_2,C_2) \cdots   W(h_n,C_n) } {\sum_{C_1, C_2, \cdots, C_n} W( h_1,C_1) W( h_2,C_2) \cdots   W(h_n,C_n) },
\end{equation}
where $W(h,C)$ corresponds to the weight for a given value of $h$ and configuration C. 
Clearly, one can sample  $P( \left[h_1,C_1\right], \left[h_2,C_2\right], \cdots, \left[h_n,C_n\right])$ by carrying out $n$ independent runs.
However, parallel tempering  includes the following   exchange step:
\begin{multline} \label{eq:exchangestep}
	\left[h_1,C_1\right], \cdots, \left[\red{h_i,C_i}\right],\left[\blue{h_{i+1},C_{i+1}}\right], \cdots, \left[h_n,C_n\right]  \; \rightarrow \\
	\left[h_1,C_1\right], \cdots, \left[\red{h_i},\blue{C_{i+1}}\right],\left[\blue{h_{i+1}},\red{C_{i}}\right], \cdots, \left[h_n,C_n\right]
\end{multline}
which, for a symmetric proposal matrix, will  be accepted with probability
\begin{equation}
	\text{ min} \left( 1,   \frac{ W(h_i,C_{i+1}) W(h_{i+1},C_{i})}{W(h_i,C_{i}) W(h_{i+1},C_{i+1})} \right).
\end{equation}
In this way a configuration can meander in parameter space $h$ and  explore regions where ergodicity is not an issue. In the context of spin-glasses, a low temperature configuration, stuck in a local minima, can heat up, overcome the potential  barrier and then cool down again. 
 
A judicious choice of the values $h_i$ is important to obtain a good acceptance rate for the exchange step.  With  $W(h,C)  = e^{- S(h,C) }$, the  distribution of the action $S$  reads:
\begin{equation}
	 \mathcal{P}( h, S ) =   \sum_{C}     P( h,C )   \delta ( S(h,C) -  S ). 
\end{equation}
A given exchange step can only be accepted if the distributions $\mathcal{P}(h,S)$ and $\mathcal{P}(h+\Delta h, S)$ overlap. For 
$\langle S \rangle_{h}  < \langle S \rangle_{h +  \Delta h} $   one can formulate this  requirement as:
\begin{equation}
	\langle S \rangle_{h}  +\langle \Delta S \rangle_{h}   \simeq \langle S \rangle_{h +  \Delta h}  - \langle \Delta S \rangle_{h + \Delta h} ,  \text{    with   }   
\langle \Delta S \rangle_{h}   =  \sqrt{ \langle \left(    S -  \langle S   \rangle_h  	\right)^2 \rangle_h} .
\end{equation}
Assuming  $ \langle \Delta S \rangle_{h + \Delta h}  \simeq \langle \Delta S \rangle_{h} $  and expanding in $\Delta h$ one obtains: 
\begin{equation}
	\Delta h \simeq \frac{ 2  \langle \Delta S \rangle_{h}    }{ \partial \langle S \rangle_{h} / \partial h}.  
\end{equation} 
The above equation becomes transparent  for  classical systems  with $ S(h,C) =  h H(C) $.  In this case, the above equation reads: 
\begin{equation}
	\Delta h       \simeq  2 h \frac{  \sqrt{c} } { c  + h \langle H \rangle_h},  \text{   with  } c = h^2    \langle \left(  H -  \langle H   \rangle_h \right)^2 \rangle_h .
\end{equation} 
Several comments are in order:
\begin{itemize}
\item[i)] Let us identify $h$ with the inverse temperature  such that $c$ corresponds to the specific heat. This quantity is extensive,  as well as the energy, such that $ \Delta h \simeq 1/{\sqrt{N}} $ where $N$ is the system size.
\item[ii)] Near a phase transition the specific heat can diverge, and $h$ must be chosen with particular care.
\item[iii)]  Since the action is formulation dependent, also the acceptance rate of the exchange move equally depend  upon the formulation. 
\end{itemize}
%\mycomment{MB: Do you track the $n-1$ exchange acceptance rates $acc(i,i+1)$ for the $n$ replicas in the code? Could the exchange rates be an efficient way to locate  a phase transition in the parameters space of $h$, without a priori knowing the order parameter? Also for topological phase transitions w/o an order parameter?  }
%\mycomment{FFA:  Yes I do track the individual acceptances and I do see  singularities in the  acceptance at the phase transition. However, owing to comment iii) at would now be very careful at interpreting the results since they are really formulation dependent. }
The quantum Monte Carlo code in the ALF project carries out parallel-tempering runs when the script \path{configure.sh} is called with the argument \path{Tempering} before compilation, see Sec.~\ref{sec:compilation}.
