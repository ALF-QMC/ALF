%-------------------------------------------------------------------------------------
\subsubsection{Hubbard  model in the canonical ensemble}
%-------------------------------------------------------------------------------------

To simulate the Hubbard model in the canonical ensemble  one can add the constraint: 
\begin{equation}
	\hat{\mathcal{H}}   = \hat{\mathcal{H}_{tU}}     + \underbrace{\lambda \left( \hat{N} -  N \right)^{2}}_{\equiv \hat{H}_\lambda }
\end{equation}
In the limit $\lambda \rightarrow \infty $, the uniform charge fluctuations,
\begin{equation} 
      S ( \vec{q} = 0)   =  \sum_{\vec{r}}   \left[ \langle \hat{n}_{\vec{r}}  \hat{n}_{\vec{0}} \rangle  - \langle \hat{n}_{\vec{r}}\rangle \langle  \hat{n}_{\vec{0}} \rangle  \right]
\end{equation} 
are suppressed and the grand-canonical simulation maps onto a canonical one.  
To implement this in the QMC code of the ALF project,  we have adopted the following strategy.   Since  $ \left( \hat{N} -  N \right)^{2}  $ effectively corresponds to a long-range interaction one may   face the issue that the acceptance rate of a single  HS flip becomes very small on large lattices. To circumvent this problem we have  used the following decomposition: 
\begin{equation}
	e^{-\beta \hat{H}}  =   \prod_{\tau = 1}^{L_{\text{Trotter}}} \left[  e^{-\Delta \tau \hat{H}_t} e^{-\Delta \tau \hat{H}_U}  
	\underbrace{e^{-\frac{\Delta \tau}{n_{\lambda}} \hat{H}_{\lambda} } \cdots e^{-\frac{\Delta \tau}{n_{\lambda}} \hat{H}_{\lambda} } }_{n_\lambda \text{-times } }\right].
\end{equation}
Thereby, we need $n_\lambda $ fields per time slice  to impose the constraint.  Since for each field the coupling  constant is suppressed by a factor $n_{\lambda}$, we can monitor the acceptance. 
An implementation of this program can be found in  \path{Prog/Hamiltonian_Hub_Canonical.f90}  and a test run in the directory \path{Examples/Hubbard_Mz_Square_Can}

