\documentclass[aps,amsmath,amssymb,nofootinbib,superscriptaddress,showpacs,floatfix,prl,onecolumn]{revtex4-1}

\usepackage{latexsym}
\usepackage{graphicx}
\usepackage{times,psfrag,subfigure}
\usepackage{enumerate}
\usepackage{amsmath}
\usepackage{dsfont}
\usepackage{dcolumn}
\usepackage{bm,bbm}       
\usepackage{color}
\usepackage{latexsym,amsmath,amssymb,bm,euscript}
\bibliographystyle{apsrev}
\usepackage{dsfont}
\usepackage{textcomp}
\usepackage{tabularx}
\usepackage{setspace}
\usepackage{ctable}
\usepackage{sidecap}
\usepackage{placeins}



\hyphenation{ALPGEN}
\hyphenation{EVTGEN}
\hyphenation{PYTHIA}

\def\re#1{(\ref{#1})}
\def\Pf{\mathop{\mathrm{Pf}}}
\def\Z{\mathds{Z}}
\def\R{\mathcal{R}}
\def\T{\mathcal{T}}
\def\C{\mathcal{C}}
\def\S{\mathcal{S}}
\def\H{\mathcal{H}}
\def\K{\mathcal{K}}
\def\U{\mathcal{U}}
\def\A{\mathcal{A}}
\def\bk{{\bf k}}
\def\br{{\bf r}}
\def\be{{\bf e}}
\def\bS{{\bf S}}
\def\inv{^{-1}}
\def\Tr{\mathop{\mathrm{Tr}}}
\def\sgn{\mathop{\textrm{sgn}}}
\newcommand{\clb}{\color{blue}}
\newcommand{\ox}{\otimes}
\newcommand{\clr}{\color{red}}
\newcommand{\clm}{\color{magenta}}
\newcommand{\clg}{\color{cyan}}
\newcommand{\up}{\uparrow}
\newcommand{\sech}{\text{sech}~}
\newcommand{\dn}{\downarrow}
\newcommand{\beq}{\begin{equation}}
\newcommand{\eeq}{\end{equation}}
\newcommand{\beqarray}{\begin{eqnarray}}
\newcommand{\eeqarray}{\end{eqnarray}}
\newcommand{\Ref}[1]{Ref.~\onlinecite{#1}}  
\newcommand{\Sec}[1]{Sec.~\ref{#1}}  
\newcommand{\eq}[1]{Eq.~(\ref{#1})}  
\newcommand{\fig}[1]{Fig.~\ref{#1}}  
\newcommand{\figs}[1]{Figs.~\ref{#1}}  
\newcommand{\bsigma}{\mbox{\boldmath$\sigma$}}

\begin{document}

\allowdisplaybreaks



\title{NOTES: Monte Carlo simulation of topological phase transition in two dimensions}

\date{\today}
  
%
%
%
%\author{Raquel Queiroz}
%\email{r.queiroz@fkf.mpg.de}
%\affiliation{Max-Planck-Institut f\"ur Festk\"orperforschung, Heisenbergstrasse 1, D-70569 Stuttgart, Germany} 
%\author{Eslam Khalaf}
%\affiliation{Max-Planck-Institut f\"ur Festk\"orperforschung, Heisenbergstrasse 1, D-70569 Stuttgart, Germany} 
%\author{Ady Stern}
%\affiliation{Department of Condensed Matter Physics, Weizmann Institute of Science, Rehovot 76100, Israel}


\date{\today}

%\pacs{03.65.vf,74.50.+r, 73.20.Fz, 73.20.-r:}


\maketitle

\section{Goal of the Project SPT}

In this project, we want to study the interaction driven reduction of the topological classification within the symmetry class A' in two dimensions $( \Z \rightarrow \Z_4 )$ without breaking any symmetries. The according code can be found in \texttt{Hamiltonian\_SPT.f90} and compiled as \emph{make SPT} with invokes the file \texttt{Compile\_SPT}. In the following, we will first discuss the physical part of the model and later also comment of the implementation.

\section{precursor}

Let's begin with the following topologically non-trivial Dirac Hamiltonian in symmetry class A'
\begin{equation}
\H=\sum_\bk \chi_\bk [t\sin k_x\gamma_1+t\sin k_y\gamma_2 + (2+\lambda+\cos k_x +\cos k_y)\gamma_3]\chi_{\bk}^\dag\label{PreHamiltonian}
\end{equation} 
where $\gamma_i$ with $i=1,...,5$ are the anticommuting Dirac matrices of dimension 4 acting on the vector $\chi_\bk=(c_{\bk\up R},c_{\bk\dn R},c_{\bk\up L},c_{\bk\dn L})^T$.
Any choice of Dirac matrices is equally fine, you can take for example $\gamma_{1,2}=\sigma_{1,2}\tau_3$, $\gamma_{3,4}=\sigma_0\tau_{1,2}$ and $\gamma_5=\sigma_3\tau_3$. Observe that $\gamma_{2;4}^\ast=-\gamma_{2;4}$ which will be assumed through the rest of these notes.

As required by the symmetry class, this model satisfies two time-reversal ($\T_1$ and $\T_2$) and one particle-hole symmetry ($\C$). In the first quantized language, these symmetries are anti-unitary (e.g. $\T_1=\K U_1$) and their unitary parts act on the Hamiltonian as $U_\alpha H^\ast(-\bk)U_\alpha^\dag = \pm H(\bk)$ with $+(-)$ refers to the TRS (PHS). Here we find $U_1=\gamma_1\gamma_4$, $U_2=\gamma_1\gamma_5$ and $U_\C=\gamma_2\gamma_3$ with $\T_1^2 = \C^2 = 1$ and $\T_1^2 = -1$. Combing the anti-unitary symmetries pairwise generates one commuting and two anti-commuting unitary symmetries, namely $R=\T_1\T_2=\gamma_4\gamma_5$ and the chiral symmetries $\S_{1;2}=\T_{1;2}\C=\gamma_{5;4}$.

\eq{PreHamiltonian} satisfies other symmetry, the four-fold rotations $C_{4}$ as $U_{C_4}H(-k_y,k_x)U_{C_4}^\dag=H(k_x,k_y)$ with $U_{C_4}=\frac{1}{\sqrt{2}}(1+\gamma_1\gamma_2)$ and the two parity symmetries $P_{x;y}$ acting as $U_{P_x}H(k_x,-k_y)U_{P_x}^\dag=H(k_x,k_y)$ with $U_{P_x}=\gamma_1\gamma_4$ or $U_{P_x}=\gamma_1\gamma_5$ and $U_{P_y}H(-k_x,k_y)U_{P_x}^\dag=H(k_x,k_y)$ with $U_{P_x}=\gamma_2\gamma_4$ or $U_{P_x}=\gamma_2\gamma_5$. The inversion symmetry is generated by applying two $C_{4}$ rotation. Combining inversion and TR or PH symmetry leads to anti-unitary symmetries which are local in $\bk$-space.

In summary, we found the following symmetries:

 \begin{tabular}{|c|c|c|c|}
 \hline 
 Trafo of $H$ & unitary part & second qu. implementation & Trafo of $i$ \\ 
 \hline 
 $U\,H(\bk)\,U^\dag = H(\bk)$ & $U=\gamma_4\gamma_5$ & $\U \chi_\bk \U^{-1} = U\chi_\bk$ & $\U i \U^{-1} = i$ \\ 
 \hline 
 $U\,H(\bk)\,U^\dag = -H(\bk)$ & $U=\left\lbrace \begin{matrix}
 \gamma_4 \\
 \gamma_5 \\
\end{matrix}  \right.$ & $\A \chi_\bk \A^{-1} = \chi^\dag_\bk U^\dag$ & $\A i \A^{-1} = -i$ \\ 
 \hline 
 $U\,H^\ast(\bk)\,U^\dag = H(\bk)$ & $U=\left\lbrace \begin{matrix}
 \gamma_2\gamma_4 \\
 \gamma_2\gamma_5 \\
\end{matrix}  \right.$ & $\A \chi_\bk \A^{-1} = U \chi_\bk $ & $\A i \A^{-1} = -i$ \\  
 \hline 
 $U\,H^\ast(\bk)\,U^\dag = -H(\bk)$ & $U=\gamma_1\gamma_3$ & $\U \chi_\bk \U^{-1} = \chi^\dag_\bk U^\dag$ & $\U i \U^{-1} = i$ \\ 
 \hline 
 \end{tabular}  

The unitary symmetries have direct consequences of the spectrum at a given point $\bk$:
\begin{itemize}
\item As $H(\bk)$ commutes with $R=\gamma_4\gamma_5$, they can simultaneously diagonalized such that $H(\bk)\Psi_\bk=E\Psi_\bk$ and $R\Psi_\bk=r\Psi_\bk$
\item $R^2=-1$ such that $r=\pm i$. Additionally, $\left[R,S_\pm\right]=\pm 2 i S_\pm$ where $S_\pm=\gamma4 \pm i \gamma_5$. $S_\pm$ acts as a raising/lowering operator for $R$. It also squares to zero, indicating some kind of fermionic nature.
\item As $S_\pm$ anti-commutes with $H(\bk)$, one of them can be use to generate a new state $S_\pm \Psi_\bk$ with the eigenvalues $-E$ and $-r$, hence they are orthogonal even for $E=0$.
\item We use one of the anti-unitary (first quantization) symmetries to generate another state orthogonal to $\Psi_\bk$ (in some sense as a generalized Kramers pair). To guarantee it's orthogonality to $S_\pm \Psi_\bk$, the operation has to commute with $R$ such that it conserves the eigenvalue $r$. These requirements are only fulfilled by $\K \gamma_1\gamma_3$ leading to the state $\gamma_1\gamma_3 \Psi^\ast_\bk$. Observe:
\begin{equation}
\Psi^\dag_\bk \gamma_1\gamma_3 \Psi^\ast_\bk = \Psi^\dag_\bk (\gamma_1\gamma_3)^\dag (\gamma_1\gamma_3)^2  \Psi^\ast_\bk = - (\gamma_1\gamma_3\Psi)^\dag_\bk  \Psi^\ast_\bk = - \Psi^\dag_\bk (\gamma_1\gamma_3 \Psi_\bk)^\ast = - \Psi^\dag_\bk \gamma_1\gamma_3 \Psi^\ast_\bk
\end{equation}
Hence we have generated to orthogonal single particle states at energy $-E$.
\end{itemize}
Summarizing the above discussion, the spectrum has to take the form $E_\bk=(E,E,-E,-E)$. If one eigenvalue at $\bk$ vanishes, the other three have to vanish as well.

If there is a Dirac node at an arbitrary $\bk$, then there are three additional one generated by the rotation symmetry (TRS and/or PHS would only generate one more node at $-\bk$). Applying parity operations guarantees the exitence of another set of four Dirac cones. Inverting this statement leads to the following conclusions:
\begin{itemize}
\item a single Dirac node has to be at the rotation-, TRS-, and parity-invariant momenta, hence a single cone can only exist at $\bk=(0,0)$ or at $\bk=(\pi,\pi)$.
\item two Dirac nodes can  additionally be located as a pair at TRS-invariant $\bk=(0,\pi)$ and at $\bk=(\pi,0)$.
\end{itemize}

\section{Lattice Hamiltonian}

The Hamiltonian in \eq{PreHamiltonian} mostly gapped, except for $\lambda\in\{-4,-2,0\}$ where semi-metals separate topological distinct insulators. It is topologically trivial for $\lambda>0$ and for $\lambda<-4$. The other two regions are non-trivial with a winding of $\pm1$. To study the topological reduction, we have to connect two sectors with a difference in the winding number of multiples of $4$. This is not possible in the current version. We can therefore either replace $\bk\rightarrow 2\bk$ or simply add three additional copies of the original Hamiltonian. Following the second path, we refine the above model to
\begin{equation}
\H=\sum_{\bk,o} \chi_{\bk,o} [t\sin k_x\gamma_1+t\sin k_y\gamma_2 + (2+\lambda+\cos k_x +\cos k_y)\gamma_3]\chi_{\bk,o}^\dag\label{hamiltonian}
\end{equation} 
with $o=1,\dots,4$. The interaction which supposedly connects different topological sectors by gaping the semi-metal without breaking the relevant symmetries is given by
\begin{eqnarray}
\H_{\mathrm{int}} &=& \frac{V}{2}\sum_{\br,\sigma'}\left( S^{x,1}_{\br ,\sigma'} S^{x,2}_{\br ,\sigma'}  + S^{y,1}_{\br ,\sigma'} S^{y,2}_{\br ,\sigma'}\right) \\
&=& \frac{V}{8}\sum_{\br,\sigma'}\left[ \left(S^{x,1}_{\br ,\sigma'} + S^{x,2}_{\br ,\sigma'}\right)^2 - \left(S^{x,1}_{\br ,\sigma'} - S^{x,2}_{\br ,\sigma'}\right)^2 + \left(S^{y,1}_{\br ,\sigma'} + S^{y,2}_{\br ,\sigma'}\right)^2 - \left(S^{y,1}_{\br ,\sigma'} - S^{y,2}_{\br ,\sigma'}\right)^2\right]
\end{eqnarray}
with $\bS_{\br ,\sigma '}^1=( \chi_{\br , 1}^\dag , \chi_{\br ,2}^\dag ) \bsigma P_{\sigma ' } \gamma_5 (\chi_{\br , 1},\chi_{\br ,2})^T$ and $\bS_{\br ,\sigma '}^2=( \chi_{\br , 3}^\dag , \chi_{\br ,4}^\dag ) \bsigma P_{\sigma ' } \gamma_5 (\chi_{\br , 3},\chi_{\br ,4})^T$ where $P_{\sigma ' }=\frac{1}{2}(1+i\sigma ` \gamma_3 \gamma_4)$ is yet another projector.
 
%Two cases can be distinguished in this Hamiltonian, $\sgn t_x=\sgn t_y$ and $\sgn t_x=-\sgn t_y$, which have winding numbers of absolute value 1 but opposite sign. Higher winding numbers can be achieved by taking larger hopping steps:
%\begin{equation}
%\H=\sum_\bk \chi_\bk [t_x\sin n_x k_x\gamma_1+t_y\sin n_y k_y\gamma_2 + m(1+\cos n_x k_x +\cos n_y k_y)]\chi_{-\bk}^\dag
%\end{equation} 
%which enlarges the unit cell and gives us a Dirac Hamiltonian with winding number given by $w=\pm (n_x+n_y)$. 
%We set all parameters to 1 except $t_y$ which we choose to be a tuning parameter $\lambda$. In this symmetry class and dimension, we want to show that there is an adiabatic path connecting the sectors $w=+4$ and $w=-4$. These winding numbers can be achieved by choosing $n_x=n_y=2$ and write the lattice Hamiltonian:
%\begin{equation}
%\H=\sum_\br [i\chi_\br \gamma_1\chi^\dag_{\br+2\be_x}+
%i\lambda \chi_\br \gamma_2\chi^\dag_{\br+2\be_y}+
%\chi_\br \gamma_3\chi^\dag_{\br}+
%\chi_\br \gamma_3\chi^\dag_{\br+2\be_x}+
%\chi_\br \gamma_3\chi^\dag_{\br+2\be_y}]+h.c.
%\end{equation}
%Here $\be_x$ and $\be_y$ are the unit vectors in the lattice, and $\lambda\in [-1,1]$. This transition leads to a gap closing for $\lambda=0$ where the invariant is changed. Both terms must vanish independently, thus the gap will close at the points $\bk=(\pm\pi/2,\pm\pi/4)$ and $\bk=(\pm\pi/2,\pm3\pi/4)$.
%In this basis, the interaction that allows for connecting these sectors is given by
%\begin{align}
%\H_{\rm int}=(\chi_\br\gamma_5\chi^\dag_{\br+\be_x})(\chi_{\br+\be_y}\gamma_5\chi^\dag_{\br+\be_x+\be_y})+h.c.
%\end{align}
%where we have used our $2\times2$ unit cell to define the different species.

\section{Symmetries and some other important aspects}

First of all, the free part (see \eqref{hamiltonian}) still fulfils all symmetries discussed in the precursor with an additional $SU(4)$ degree of freedom that rotates in the sub-lattice/orbital space $o$.

(Raquel)
The above Hamiltonian is composed of four disconnected sectors, each describing a lattice Dirac Hamiltonian of winding number 1. The symmetries that rotate within this space is not particularly relevent for the topology, but you are welcome to use them to simplify the diagrams.

For all purposes, we can look at the system as complex fermions and not Majorana states, to avoid problems with particle conservation. In this case, chiral symmetry must be implemented as a sublattice symmetry. That is, we introduce an isospin degree of freedom, let us call it $L,R$ since we are in the chiral basis $S\gamma_5$ is diagonal. The Hamiltonian will then acts on the multiplet $\chi_\bk=(c_{\bk\up R},c_{\bk\dn R},c_{\bk\up L},c_{\bk\dn L})^T$. The "non-trivial" unitary symmetry we have arises because we have not used all five Dirac matrices. That is $\{H,\gamma_4\}=0$ and consequently $[H,\gamma_4\gamma_5]=0$. Let us then define $R=\gamma_4\gamma_5$. A nice description for these symmetries (but only in terms of the edge theory) is found on our paper section "from one to two dimensions". I will try finding a time during today to put this here more explicitly. 

[TBC]

\section{Sign Problem}

The basic object of interested for Monte-Carlo-Simulations is the partition function $Z=\mathrm{Tr}\left[e^{-\beta \H }\right]$. In many versions of QMC, one has to use a Trotter decomposition with $\Delta \tau=\frac{\beta}{N_t}$ such that $Z=\mathrm{Tr}\left[\prod^{N_t}\left(e^{- \Delta\tau \H }\right)\right]$. Allowing a systematic error $\mathcal{O}({\Delta \tau}^2)$, we can now use $e^{\Delta \tau(A+B)}=e^{\Delta \tau A}e^{\Delta \tau B}$, even if $A$ and $B$ do not commute.

For auxiliary field Methods, we use a Hubbard-Stratonovic-Transformation to reduce the interaction terms in $\H$ to quadratic order in the fermion fields at the expense of newly introduced fields $\Phi_n(i,\tau)$. Here $i$ labels the real space position and $\tau$ the "time-slice" (the specific factor of the Trottor decomposition) whereas $n$ allows for more than one transformation.

For the above Hamiltonian, it is useful to introduce the four atomic unit cell more carefully. $I$ labels the position in the new lattice, the orbital $O \in \left\lbrace A,B,C,D \right\rbrace$ classifies the "old positions" within a unit cell as:
\begin{subequations}
\begin{eqnarray}
\chi_{I,A} &=& \chi_{\br =2I} \\
\chi_{I,B} &=& \chi_{\br =2I+\be_x} \\
\chi_{I,C} &=& \chi_{\br =2I+\be_y} \\
\chi_{I,D} &=& \chi_{\br =2I+\be_x+\be_y}
\end{eqnarray}
\end{subequations}
Accordingly, $\H_{\rm int}$ takes the form:
\begin{eqnarray}
\H_{\rm int} &=& \sum_I (\chi_{I,A} \gamma_5 \chi^\dag_{I,B})(\chi_{I,C} \gamma_5 \chi^\dag_{I,D} )+h.c. \\
&=& \frac{1}{4}\sum_I \left[(\chi_{I,A} \gamma_5 \chi^\dag_{I,B})+(\chi_{I,C} \gamma_5 \chi^\dag_{I,D} )\right]^2 - \left[(\chi_{I,A} \gamma_5 \chi^\dag_{I,B})-(\chi_{I,C} \gamma_5 \chi^\dag_{I,D} )\right]^2+h.c. \label{beforeHST}
\end{eqnarray}

At this stage, we are prepared to perform the Hubbard-Stratonovic-Transformations, introducing real auxiliary filed $\Phi_e(I,\tau)$ for $\left[ (\chi_{I,A} \gamma_5 \chi^\dag_{I,B}) + (\chi_{I,C} \gamma_5 \chi^\dag_{I,D}) \right]$ and $\Phi_o(I,\tau)$ for $\left[(\chi_{I,A} \gamma_5 \chi^\dag_{I,B})-(\chi_{I,C} \gamma_5 \chi^\dag_{I,D} )\right]$, resulting in
\begin{eqnarray}
Z &=& \int D\left\lbrace \Phi_e(I,\tau),\Phi_o(I,\tau) \right\rbrace \exp\left[ -\Delta \tau \sum_{I,\tau} \left(  \Phi_e(I,\tau)^2 + \Phi_o(I,\tau)^2  \right) \right] \\
&&\times\mathrm{Tr}\left[\prod^{N_t}_{\tau=1} e^{-\Delta\tau\left(\H_0 + i\Phi_e(I,\tau)\left[ (\chi_{I,A} \gamma_5 \chi^\dag_{I,B}) + (\chi_{I,C} \gamma_5 \chi^\dag_{I,D}) \right] + \Phi_o(I,\tau) \left[(\chi_{I,A} \gamma_5 \chi^\dag_{I,B})-(\chi_{I,C} \gamma_5 \chi^\dag_{I,D} )\right] \right)} \right].
\end{eqnarray}
The relative phase $i$ in the $\Phi$-linear terms is necessary to take care of the relative minus in Eq.\eqref{beforeHST}. Changing $\H_{int} \rightarrow -\H_{int}$ reverses the phase $i$ to $\Phi_o$. The "+h.c." can be ignored as the given operator in Eq.\eqref{beforeHST} is already hermitian, given the relation between $\chi$ and $c$.

For the Monte-Carlo sampling, we have to guaranty a non-negative values of the trace in the Fock space. One sufficient way to do this uses a anit-unitary symmetry $T$ which squares $T^2=-1$ and commutes with the operator form which the trace is taken. In this case, there exist a time-reversal partner to every eigenstate of the operator whose eigenvalues are complex conjugate pairs.

One possible choice for this "time-reversal" symmetry is given by 
\begin{equation} \label{TRS}
T(\chi^\dag_{I,A}, \chi^\dag_{I,B}, \chi^\dag_{I,C}, \chi^\dag_{I,D})T^{-1} = (-\chi^\dag_{I,C}, \chi^\dag_{I,D}, \chi^\dag_{I,A}, -\chi^\dag_{I,B}),
\end{equation}
which squares to $-1$, by definition anti-unitary and transforms $ (\chi_{I,A} \gamma_5 \chi^\dag_{I,B})  \leftrightarrow - (\chi_{I,C} \gamma_5 \chi^\dag_{I,D} )$ into each other while changing the sign. Accordingly, it commutes with the $\Phi$-linear terms. This symmetry also commutes with $\H_0$ if it is real in real space, in other words, the "old" time-reversal symmetry of $\H_0$ should be $T=K$. This way it would square to $1$ which influence the symmetry class.

Alternativly, we could remove all minus signs in Eq. \eqref{TRS} and add therefore a unitary part $U$ for the four "original" Dirac degrees of freedom such that $\left\lbrace \gamma_5 , KU \right\rbrace = 0$ with $(KU)^2=-1$. I'm not so sure about this version yet, but I guess it should work.

 \end{document}



