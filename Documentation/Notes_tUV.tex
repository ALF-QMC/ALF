\documentclass[10pt,Arial]{scrartcl}
\usepackage{graphicx}
\usepackage[margin=2.5cm]{geometry}
\usepackage[utf8]{inputenc}
\usepackage{bbm}            
\usepackage[fleqn]{amsmath} 
\usepackage{amssymb}
\usepackage{wasysym}
\usepackage{color}
\usepackage{soul}
\usepackage{xspace}
\usepackage{bm}
\usepackage{subfigure}
\usepackage{tcolorbox}
\usepackage{dsfont}
\usepackage{footnote}
\usepackage{alltt}
\usepackage{xcolor}
\usepackage{lmodern}
\usepackage{listings}
\usepackage{url}
\usepackage{booktabs}
\usepackage{hyperref} 

\definecolor{light-gray}{gray}{0.95}
\definecolor{light-gray2}{gray}{0.88}
\definecolor{comment-color}{rgb}{0.8,0.1,0.1}
\definecolor{keyword-color}{rgb}{0.3,0.3,1}
\sethlcolor{light-gray2}
\newcommand*{\red}{\textcolor{red}}
\newcommand*{\blue}{\textcolor{blue}}

\lstdefinestyle{fortran}{
  language={[03]Fortran},
  basicstyle=\ttfamily\footnotesize,
  keywordstyle=\color{keyword-color},
  commentstyle=\color{comment-color},
  morecomment=[l]{!\ },% Otherwise it's a comment only with space after !
  breakatwhitespace=false,
  keepspaces=true,
  showstringspaces=false,
  columns=fullflexible,
  backgroundcolor=\color{light-gray},
  frame=single,
  xleftmargin=3.4pt,
  xrightmargin=3.4pt
}

\newcommand{\ve}[1]{\boldsymbol{#1}}
\def\Tr{\mathop{\mathrm{Tr}}}
\def\Trf{\mathop{\mathrm{Tr}_{\mathrm{F}}}}
\makesavenoteenv{tabular}
\makesavenoteenv{table}
\setkomafont{author}{\large}
\setkomafont{date}{\large}

\newcommand{\mycomment}[1]{{\color{blue} #1}}
\newcommand{\FAcomment}[1]{{\color{red} #1}}


\begin{document}
%---------------------------------------------------------------------------------------------------------
\title{ Extended Hubbard model: 
\texttt{ Hamiltonian\_extended\_Hubbard\_smod.F90 } }
\subtitle{}
\author{}
\date{}
%---------------------------------------------------------------------------------------------------------
\maketitle 
The  Hamiltonian  that  we  will  consider here  is  given  by: 
\begin{equation}
\label{eqn:Ham_Ext_Hub}
	\hat{H}  =       \sum_{ b =  \left< i,j \right>  } \sum_{\sigma=1}^{N}  \left(  
  \hat{c}^{\dagger}_{i,\sigma} T^{\sigma}_{i,j}    \hat{c}^{\dagger}_{j,\sigma}  +  \text{H.c.} \right)
  +  \sum_{ b=\left< i,j \right> }  
  \frac{V_{i,j}}{2N} \left( [n_{i} - N/2]  +  s_V [ n_{j}  - N/2 ]  \right)^2   +  
  \sum_{i} \frac{U_i}{N} \left( n_i -  \frac{N}{2} \right)^2
\end{equation}
The  implementation supports all  standard ALF lattices and  the  hopping as  well as  the 
interaction $V$  are  restricted to nearest neighbors.    The parameter file  for  this 
specific model reads: 
\begin{lstlisting}[style=fortran,escapechar=\#,breaklines=true]
&VAR_extended_Hubbard      !! Variables for the Extended Hubbard
ham_T      = 1.d0          ! Hopping parameter
Ham_chem   = 0.d0          ! Chemical potential
Ham_U      = 1.d0          ! Hubbard interaction
Ham_V1     = 0.d0          ! nearest neighbor interaction
ham_T2     = 1.d0          ! For bilayer systems
Ham_U2     = 0.d0          ! For bilayer systems
Ham_V2     = 0.d0          ! For bilayer systems
ham_Tperp  = 0.d0          ! For bilayer systems
Ham_Vperp  = 0.d0          ! For bilayer systems
Ham_SV     = 1.0           ! = +/- 1: Sign convention for interaction
/
\end{lstlisting}
In the above \texttt{Ham\_T, Ham\_V, Ham\_U} are the    nearest  neighbor  hopping, nearest  neighbor  interaction  and Hubbard repulsion on the the first layer.   \texttt{Ham\_T2, Ham\_V2, Ham\_U2}   are  the corresponding quantities  on the second layer.  \texttt{Ham\_Tperp, Ham\_Vperp}   define  the inter-layer couplings.  Finally $ Ham\_chem $  is  the chemical potential.   
To use  this  Hamiltonian  you  have to specify: 
\begin{lstlisting}[style=fortran,escapechar=\#,breaklines=true]
&VAR_ham_name
ham_name = "extended_Hubbard"
/
\end{lstlisting}
in the \texttt{parameters}  file.

In this formulation,  there is  some   \textit{  double counting  }  of  the  Hubbard  term.  In particular  expanding  the  square  $V$-term  gives  the  Hamiltonian:  
\begin{equation}
  \label{eqn:Ham_Ext_Hub_0}
    \hat{H}  =       \sum_{b=\langle  i,j \rangle } \sum_{\sigma=1}^{N}   
    \hat{c}^{\dagger}_{i,\sigma} T^{\sigma}_{i,j}    \hat{c}^{\dagger}_{j,\sigma}  + 
    \sum_{b=\langle  i,j \rangle }  
    \frac{s_V V_{i,j}}{N} \left( [n_{i} - N/2] \right) \left([ n_{j}  - N/2 ]  \right)   +  
    \sum_{i}  \frac{U^{eff}_i}{N} \left( n_i -  \frac{N}{2} \right)^2
  \end{equation}
  In the  above   
  \begin{equation}
    U^{eff}_{i}    = U_i +  \frac{1}{2}\sum_{b = (n,m)} V_b  \left( \delta_{i,m} +  s_V^2\delta_{i,n}\right).
  \end{equation}
  For  the  single  layer  lattices  with uniform $U$ and $V$, and  $s_V = 1$,  $U^{eff} =   U  + Z V $ 
  where  $Z=4$ ($Z=3$)   for  the square (honeycomb)  lattice.  

  Note that  the potential and  total  energies  are   defined  as in the  Hamiltonian.   That is   the file  \texttt{Ener\_scalJ}   corresponds to  
  $ \langle \hat{H}   \rangle $.

\end{document}