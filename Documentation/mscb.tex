% Copyright (c) 2021 The ALF project.
% This is a part of the ALF project documentation.
% The ALF project documentation by the ALF contributors is licensed
% under a Creative Commons Attribution-ShareAlike 4.0 International License.
% For the licensing details of the documentation see license.CCBYSA.

% !TEX root = doc.tex

%-------------------------------------------------------------------------------------
\section{The Minimum Split Checkerboard Decomposition}
%-------------------------------------------------------------------------------------
ALF has a module for the automatic decomposition of arbitrary graphs/ hermitian matrices into 
an almost minimal number of so-called strictly sparse matrices. We call a hermitian matrix $A$
strictly sparse if it has the property
\begin{equation}
A_{i,j} = 0 \lrarrow \left( e^A \right)_{i,j} = 0 \forall i,j
\label{eq:strictsparse}
\end{equation}
To exploit that property for a mtrix, \eg the hopping matrix $T$
we need a decomposition of $T$ in the form
\begin{equation}
T=\sum_i^N T^i
\label{eq:sep}
\end{equation}
with each $T^i$ having the property \eqref{strictsparse}, because then by an application
of the Lie-Trotter formula a first order approximation of $e^{\Delta \tau T}$ can be constructed.
Of course the question remains how to find a decomposition \eqref{eq:sep}.
For some lattices you have seen an example of this type of decomposition in Figure \ref{fig_predefined_lattices}.
As you might notice from the figure, all edges that are attached to a vertex have a different color.
And indeed, as pointed out by \cite{Lee2013} a color decomposition will give a decomposition of $T$
with the required properties. More generally the problem of coloring the edges of a graph
is known as the graph-coloring problem and the number of colors required for a graph $G$ is known as 
its chromaticity index $\chi(G)$. An important structural result is Vizing's theorem
which states that a graph $G$ of degree $g$ (meaning the maximum of the number of edges over $G$) the following holds
\begin{equation}
g \leq \chi(G) \leq g+1
\end{equation}
thereby enabling a classification of graphs into two classes.
Determining the chromaticity index for an arbitrary $G$ is an NP-hard problem.
ALF does not try to solve this challenge but tries to find a decomposition with the help of the
Misra-van-Gries graph coloring algorithm.
This algorithm allows itself the freedom to utilize $g+1$ colors, from which it can derive the important
knowledge that at every vertex it encounters during the process it has always one free color.
In ALF we allow to utilize higher-order checkerboard (as e.g. demonstrated in the appendix of \cite{goth2020})
via a config switch.
